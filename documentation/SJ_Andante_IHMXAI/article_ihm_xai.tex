%%
%% This is file `sample-manuscript.tex',
%% generated with the docstrip utility.
%%
%% The original source files were:
%%
%% samples.dtx  (with options: `manuscript')
%% 
%% IMPORTANT NOTICE:
%% 
%% For the copyright see the source file.
%% 
%% Any modified versions of this file must be renamed
%% with new filenames distinct from sample-manuscript.tex.
%% 
%% For distribution of the original source see the terms
%% for copying and modification in the file samples.dtx.
%% 
%% This generated file may be distributed as long as the
%% original source files, as listed above, are part of the
%% same distribution. (The sources need not necessarily be
%% in the same archive or directory.)
%%
%% The first command in your LaTeX source must be the \documentclass command.


%% FORMAT WITH 1 OR 2 COLUMNS %%%%%%%%%%%%%%%%%%%%%%%%%%%%%%%%%%%%%%%%%%%%%%%%

%% Pour la soumission en relecture (une seule colonne)
\documentclass[manuscript,authordraft]{acmart}

%% Pour publication de la version acceptee (deux colonnes)
%\documentclass[sigconf]{acmart}

%%%%%%%%%%%%%%%%%%%%%%%%%%%%%%%%%%%%%%%%%%%%%%%%%%%%%%%%%%%%%%%%%%%%%%%%%%%%%%

%\usepackage{eurosym}  %% Fournit \euro{} si vous avez besoin du symbole Euro
\graphicspath{{img/}}  %% Le répertoire pour vos images
\def\venue{IHM~'22} %% Le nom abrégé de la conférence

%% handling bilingual document %%%%%%%%%%%%%%%%%%%%%%%%%%%%%%%%%%%%%%%%%%%%%%%

\usepackage[french,english]{babel}
\newcommand{\en}[1]{\foreignlanguage{english}{#1}}
\newcommand{\fr}[1]{\foreignlanguage{french}{#1}}

%% handling accents by JMJ
\usepackage[utf8]{inputenc}

%%%%%%%%%%%%%%%%%%%%%%%%%%%%%%%%%%%%%%%%%%%%%%%%%%%%%%%%%%%%%%%%%%%%%%%%%%%%%%


%%
%% \BibTeX command to typeset BibTeX logo in the docs
\AtBeginDocument{%
  \providecommand\BibTeX{{%
    \normalfont B\kern-0.5em{\scshape i\kern-0.25em b}\kern-0.8em\TeX}}}

%% Rights management information.  This information is sent to you
%% when you complete the rights form.  These commands have SAMPLE
%% values in them; it is your responsibility as an author to replace
%% the commands and values with those provided to you when you
%% complete the rights form.
%\setcopyright{none}
\setcopyright{acmcopyright}
%\setcopyright{acmlicensed}
%\setcopyright{rightsretained}
%\setcopyright{usgov}
%\setcopyright{usgovmixed}
%\setcopyright{cagov}
%\setcopyright{cagovmixed}
\copyrightyear{2022}
\acmYear{2022}
\acmDOI{10.1145/XXXXXXXXX.XXXXXXXX}

%% These commands are for a PROCEEDINGS abstract or paper.
\acmConference[\venue{}]{33\textsuperscript{e} conférence Francophone
	sur l'Interaction Humain-Machine}{April 05--08, 2022}{Namur, Belgique}
\acmBooktitle{\venue{}~: 33\textsuperscript{e} conférence Francophone
	sur l'Interaction Humain-Machine, April 05--08, 2022, Namur, Belgique}
\acmPrice{15.00}
\acmISBN{XXX-X-XXXX-XXXX-X/XX/XX}


%%
%% Submission ID.
%% Use this when submitting an article to a sponsored event. You'll
%% receive a unique submission ID from the organizers
%% of the event, and this ID should be used as the parameter to this command.
%%\acmSubmissionID{123-A56-BU3}

%%
%% The majority of ACM publications use numbered citations and
%% references.  The command \citestyle{authoryear} switches to the
%% "author year" style.
%%
%% If you are preparing content for an event
%% sponsored by ACM SIGGRAPH, you must use the "author year" style of
%% citations and references.
%% Uncommenting
%% the next command will enable that style.
%%\citestyle{acmauthoryear}

%%
%% end of the preamble, start of the body of the document source.
\begin{document}
\selectlanguage{english}
\sloppy

%%
%% TITRE : OBLIGATOIRE en français ET en anglais
\def\titreFR{Sur l'explicitation de l'apprentissage inductif logique avec le notebook Andante}
\def\titreEN{Towards Exploring Inductive Logic Learning with the Andante Notebook}
\def\titreSHORT{ILP et Andante} %% Si le titre FRANCAIS de l'article est trop long pour les entêtes ou pieds des pages

% \def\titreEN{The Title of the Paper: Template of Papers in English for \venue{} Conference}
% \def\titreFR{Le titre de l'article : Gabarit des articles en anglais pour \venue{}}
% \def\titreSHORT{Template for \venue{}} %% Si le titre ANGLAIS de l'article est trop long pour les entêtes ou pieds des pages

%%
%% The "title" command has an optional parameter,
%% allowing the author to define a "short title" to be used in page headers.
\title[\en{\titreSHORT}]{\en{\titreEN}}
\subtitle{\fr{\titreFR}}

%%
%% The "author" command and its associated commands are used to define
%% the authors and their affiliations.
%% Of note is the shared affiliation of the first two authors, and the
%% "authornote" and "authornotemark" commands
%% used to denote shared contribution to the research.

\author{Simon Jacquet}
\email{simon.jacquet@unamur.be}
% \affiliation{%
%   \institution{Institut de recherche NADI, Université de Namur}
%   \city{Namur}
%   \country{Belgique}
% }

\author{Sarah Pinon}
\email{sarah.pinon@unamur.be}
% \affiliation{%
%   \institution{Institut de recherche NADI, Université de Namur}
%   \city{Namur}
%   \country{Belgique}
% }

\author{Jean-Marie Jacquet}
\email{jean-marie.jacquet@unamur.be}
% \affiliation{%
%   \institution{Institut de recherche NADI, Université de Namur}
%   \city{Namur}
%   \country{Belgique}
% }

\author{Isabelle Linden}
\email{isabelle.linden@unamur.be}
% \affiliation{%
%   \institution{Institut de recherche NADI, Université de Namur}
%   \city{Namur}
%   \country{Belgique}
% }

\author{Wim Vanhoof}
\email{wim.vanhoof@unamur.be}
\affiliation{%
  \institution{NADI Research Institute, University of Namur}
  \city{Namur}
  \country{Belgium}
}

% \author{Dominique Dupond}
% \authornote{Both authors contributed equally to this research.}
% \email{d.dupond@institut-ihm.fr}
% \affiliation{%
%   \institution{Institut de l'IHM}
%   \city{Ville}
%   \country{France}
% }
% 
% \author{Claude Dupont}
% \authornotemark[1]
% \email{c.dupont@universite-ihm.ca}
% \affiliation{%
%   \institution{Université de l'IHM}
%   \city{Super Ville}
%   \state{Super Province}
%   \country{Canada}
% }
% 
% \author{Camille Anova}
% \email{camille@hci-corporation.ch}
% 
% \author{Charlie Goms}
% \email{charlie@hci-consortium.ch}
% \affiliation{%
% 	\institution{HCI Corporation}
% 	\city{Green Town}
% 	\country{Suisse}
% }

%%
%% By default, the full list of authors will be used in the page
%% headers. Often, this list is too long, and will overlap
%% other information printed in the page headers. This command allows
%% the author to define a more concise list
%% of authors' names for this purpose.

%% Si la liste des auteurs est trop longue en entête ou pied de page
\renewcommand{\shortauthors}{S. Jacquet, et al.}

%%
%% The abstract is a short summary of the work to be presented in the
%% article.
\begin{abstract} %% OBLIGATOIRE
  \en{ Inductive logic programming is a form of machine learning which
    consists in inducing logical formulas from a background theory as
    well as positive and negative examples.  Many systems have been
    proposed, of which the most famous are FOIL, Progol, Aleph and
    Metagol (see \cite{CrDu20} for a description). If the results
    produced are explainable because of their declarative character,
    the fine understanding of the learning process as well as the
    models produced raises many questions of intelligibility.  The notebook Andante has been designed to tackle these issues. More precisely, in order to allow a
    user to analyze finely both the learning process as well as the
    models produced and, consequently, to increase the
    interpretability of the models generated, we have built a notebook
    offering the following functionalities: }
%
(i) the lecture of programs in tailored execution contexts,
(ii) the execution of Prolog queries allowing to test the code
  introduced but also different working hypotheses,
(iii) the definition of high level auxiliary concepts by means of
  suitable predicates, reusable in the learning process, 
(iv) the generation of models induced by different methods and
  the inspection of intermediate results produced during this generation.
%  
\end{abstract}

%% french abstract %%%%%%%%%%%%%%%%%%%%%%%%%%%%%%%%%%%%%%%%%%%%%%%%%%%%%%%%%%
\begin{resume} %% OBLIGATOIRE
\fr{La programmation logique inductive consiste à apprendre des
  formules logiques à partir d'une théorie de base et d'exemples
  positifs et négatifs. De nombreux systèmes ont été proposés dont les
  plus connus sont FOIL, Progol, Aleph et Metagol (cf \cite{CrDu20}
  pour une description). Si les résultats produits sont explicables de
  part leur caractère déclaratif, la compréhension fine du processus
  d'apprentissage ainsi que des modèles produits soulève néanmoins de
  nombreuses questions d'intelligibilité.  C'est aux fins de pouvoir
  répondre à ces questions qu'a été développé le notebook
  Andante. Plus précisément, de manière à permettre à un utilisateur
  d'analyser finement le processus d'apprentissage ainsi que les
  modèles produits et, par suite d'augmenter l'interprétabilité des
  modèles générés, nous avons construit un notebook offrant les
  fonctionnalités suivantes :
%
(i) la lecture de programmes dans des contextes d'exécution personnalisés,
(ii) l'exécution de requêtes Prolog permettant de tester le code
  introduit mais aussi différentes hypothèses de travail,
(iii) la définition de concepts auxiliaires de haut niveau par le
  biais de prédicats adaptés, réutilisables dans le processus
  d'apprentissage,
(iv) la génération de modèles induits par différentes méthodes et
  l'inspection de résultats intermédiaires produits lors de cette génération.
%
}
\end{resume}
%%%%%%%%%%%%%%%%%%%%%%%%%%%%%%%%%%%%%%%%%%%%%%%%%%%%%%%%%%%%%%%%%%%%%%%%%%%%%


%%
%% Keywords. The author(s) should pick words that accurately describe
%% the work being presented. Separate the keywords with commas.
% \keywords{\en{Interaction technique, Models}} %% OBLIGATOIRE
\keywords{\en{Explainability, Inductive Logic Programming}} %% OBLIGATOIRE

%% french keywords %%%%%%%%%%%%%%%%%%%%%%%%%%%%%%%%%%%%%%%%%%%%%%%%%%%%%%%%%%
% \motscles{\fr{Techniques d'interaction, Modèles}} %% OBLIGATOIRE
\motscles{\fr{Explicabilité, Programmation logique inductive}} %% OBLIGATOIRE
%%%%%%%%%%%%%%%%%%%%%%%%%%%%%%%%%%%%%%%%%%%%%%%%%%%%%%%%%%%%%%%%%%%%%%%%%%%%%


%%
%% This command processes the author and affiliation and title
%% information and builds the first part of the formatted document.
\maketitle

%%
%% The next two lines define the bibliography style to be used, and
%% the bibliography file.
\bibliographystyle{ACM-Reference-Format}
\bibliography{ihm_xai.bib}

\end{document}
\endinput
%%
%% End of file `sample-manuscript.tex'.
